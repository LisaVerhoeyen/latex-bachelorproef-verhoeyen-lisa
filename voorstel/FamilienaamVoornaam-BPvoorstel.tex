%==============================================================================
% Sjabloon onderzoeksvoorstel bachproef
%==============================================================================
% Gebaseerd op document class `hogent-article'
% zie <https://github.com/HoGentTIN/latex-hogent-article>

% Voor een voorstel in het Engels: voeg de documentclass-optie [english] toe.
% Let op: kan enkel na toestemming van de bachelorproefcoördinator!
\documentclass{hogent-article}
\usepackage{xcolor, soul}

% Invoegen bibliografiebestand
\usepackage[backend=biber,style=apa]{biblatex}
\DeclareLanguageMapping{dutch}{dutch-apa}
\addbibresource{voorstel.bib}

% Informatie over de opleiding, het vak en soort opdracht
\studyprogramme{Professionele bachelor toegepaste informatica}
\course{Bachelorproef}
\assignmenttype{Onderzoeksvoorstel}
% Voor een voorstel in het Engels, haal de volgende 3 regels uit commentaar
% \studyprogramme{Bachelor of applied information technology}
% \course{Bachelor thesis}
% \assignmenttype{Research proposal}

\academicyear{2024-2025}

\title{Stemmen onderscheiden voor het opsporen van elderspeak in de Vlaamse zorgcontext}

\author{Lisa Verhoeyen}
\email{lisa.verhoeyen@student.hogent.be}

% TODO: Geef de co-promotor op
\supervisor[Co-promotor]{J. Campens (HOGENT, \href{mailto:jorrit.campens@hogent.be}{jorrit.campens@hogent.be})}

% Binnen welke specialisatierichting uit 3TI situeert dit onderzoek zich?
% Kies uit deze lijst:
%
% - Mobile \& Enterprise development
% - AI \& Data Engineering
% - Functional \& Business Analysis
% - System \& Network Administrator
% - Mainframe Expert
% - Als het onderzoek niet past binnen een van deze domeinen specifieer je deze
%   zelf
%
\specialisation{AI \& Data Engineering}
\keywords{stemmen onderscheiden, speech-to-text, secondary babytalk}

\begin{document}

\begin{abstract}
	Ondanks het feit dat de zorgsector de dag van vandaag vooral op persoonsgerichte zorgverlening focust, ervaren oudere patiënten vaak dat de communicatie te betuttelend is. Dit fenomeen staat bekend als elderspeak of secondary babytalk. Om dit tegen te gaan is er vanuit de opleiding verpleegkunde aan HOGENT het initiatief genomen om een applicatie te gebruiken die de studenten erop wijst als ze gebruik maken van elderspeak. Deze applicatie werd reeds ontworpen in voorgaande bachelorproeven, maar er ontbreken nog een aantal functionaliteiten voordat deze gebruikt kan worden in de opleiding. Deze paper heeft als doel de mogelijkheid te bieden om in de applicatie enkel de stem van de zorgverlener te analyseren en signalen van secondary babytalk op te sporen. Dit wordt bereikt aan de hand van het zoeken naar en trainen van geschikte modellen die stemmen kunnen onderscheiden. Het meest geschikte model wordt nadien geïmplementeerd in de applicatie. Het beoogde resultaat van deze paper is een volledig getraind model dat succesvol stemmen kan onderscheiden en geïntegreerd is in de applicatie.
\end{abstract}

\tableofcontents

% De hoofdtekst van het voorstel zit in een apart bestand, zodat het makkelijk
% kan opgenomen worden in de bijlagen van de bachelorproef zelf.
%---------- Inleiding ---------------------------------------------------------

% TODO: Is dit voorstel gebaseerd op een paper van Research Methods die je
% vorig jaar hebt ingediend? Heb je daarbij eventueel samengewerkt met een
% andere student?
% Zo ja, haal dan de tekst hieronder uit commentaar en pas aan.

%\paragraph{Opmerking}

% Dit voorstel is gebaseerd op het onderzoeksvoorstel dat werd geschreven in het
% kader van het vak Research Methods dat ik (vorig/dit) academiejaar heb
% uitgewerkt (met medesturent VOORNAAM NAAM als mede-auteur).
% 

%TODO nalezen
\section{Inleiding}%
\label{sec:inleiding}

De dag van vandaag focust de zorg- en welzijnssector vooral op persoonsgerichte zorgverlening. Dit impliceert dat de zorgverleners hun communicatiestijl aanpassen aan de zorgvragers. In de context van zorgverlening bij ouderen wordt de communicatiestijl echter vaak als te betuttelend ervaren. Men noemt dit fenomeen secondary babytalk of elderspeak. Om dit te proberen verhelpen werd er in 3 bachelorproeven van voorgaande jaren een applicatie ontwikkeld die secondary babytalk kan herkennen aan de hand van kenmerken zoals toonhoogte en stemvolume \autocite{Govaerts2022,Gussem2022,Daems2023}. In 3 hierop volgende bachelorproeven werd aangetoond dat de accuraatheid van bestaande individuele modellen echter onvoldoende blijkt te zijn om gesproken Nederlands kwaliteitsvol om te zetten naar in tekst en dus te integreren in de huidige applicatie \autocite{Branden2024,Coetsiers2024,Schryver2024}.

Deze bachelorproef zal focussen op het onderscheiden van verschillende stemmen. Dit zou ervoor moeten zorgen dat een onderscheid kan gemaakt worden tussen de stem van de zorgverlener en deze van de zorgvrager, om zo enkel de stem van de zorgverlener om te zetten naar geschreven tekst. Dit heeft als doel gebruikt kunnen worden in de opleiding van zorgverleners om aan te tonen waar ze gebruik gemaakt hebben van secondary babytalk.

%---------- Stand van zaken ---------------------------------------------------

%TODO verdere subsecties maken en literatuur opzoeken
\section{Literatuurstudie}%
\label{sec:literatuurstudie}

\subsection{Voorgaand onderzoek in verband met de applicatie}%

In voorgaande bachelorproeven werd reeds een applicatie ontwikkeld met als doel deze te gebruiken in de opleiding verpleegkunde. Volgens \textcite{Govaerts2022} is het detecteren van elderspeak mogelijk met behulp van PRAAT en Natural Language Processing (NLP). Praat is een computerprogramma voor het analyseren, synthetiseren en manipuleren van spraak en werd ontwikkeld sinds 1992 door Paul Boersma en David Weenink \autocite{Govaerts2022}.

Datzelfde jaar werd door \textcite{Gussem2022} geconcludeerd dat elderspeak onrechtstreeks gedetecteerd kan worden met AI. De software die hiervoor beschikbaar is, is echter gelimiteerd en niet altijd accuraat \autocite{Gussem2022}.

Later onderzocht \textcite{Daems2023} hoe de detectie van elderspeech verbeterd kan worden door het toepassen van een ruisfilter en een stiltefilter. De conclusie omtrent de ruisfilter was dat deze zeker helpt, maar niet alle ruis uit het signaal kan verwijderen \autocite{Daems2023}. Om dit te verbeteren opperde \textcite{Daems2023} dat het gebruik van meerdere microfoons zou kunnen helpen.
Wanneer het gaat over de stiltefilter, besloot \textcite{Daems2023} dat deze succesvol de stiltes kan detecteren en analyseren, wat het mogelijk maakt om stiltes te verwijderen en andere parameters te verbeteren. \textcite{Daems2023} Concludeerde echter ook dat factoren zoals geluidskwaliteit, gesproken dialect, linguïstische verschillen en het doel voor de spraakherkenning de kwaliteit ervan ook sterk beïnvloeden.

\textcite{Branden2024} zocht naar verdere mogelijkheden om de accuraatheid van elderspeak detectie te verhogen. Uit dit onderzoek kon geconcludeerd worden dat het gebruik van een POS-tagger ervoor zorgde dat de accuraatheid significant verhoogd werd, namelijk van 92.71\% naar 99.48\% \autocite{Branden2024}. POS-tagging is een NLP techniek waarbij woordsoorten worden toegekend aan woorden of tokens en hierbij ook rekening houdt met de context \autocite{Branden2024}.

Het onderzoek van \textcite{Schryver2024} had als doel om spraak naar tekst om te zetten. Hieruit moest echter de conclusie getrokken worden dat het accuraat omzetten van gesproken naar geschreven tekst nog niet mogelijk is \autocite{Schryver2024}. \textcite{Schryver2024} Heeft ook onderzocht of er mogelijkheden zijn om de performantie van de applicatie te kunnen verbeteren, maar de conclusie was dat dit niet het geval lijkt te zijn.

\subsection{Onderzoek het onderscheiden van stemmen}

Er zijn al verschillende onderzoeken gebeurd om stemmen te onderscheiden in audio fragmenten. \textcite{Zeghidour2021} hebben een model ontworpen dat gebruik maakt van 2 convolutionele subnetwerken om zo succesvol de verschillende signalen van elkaar te onderscheiden. Het eerste subnetwerk is een 'speaker stack', die de input in kaart brengt aan de hand van een set vectoren. Het tweede subnetwerk is een 'separation stack', die gebruik maakt van de input en de vector representaties. De combinatie van deze 2 subnetwerken zorgt ervoor dat het model een representatie van elke bron maakt en op basis hiervan een schatting maakt van de gescheiden signalen. \autocite{Zeghidour2021}

Een later onderzoek werd gedaan door \textcite{Nachmani2020}, waar ze geprobeerd hebben om het 'cocktail party probleem' op te lossen. Het 'cocktailparty probleem' is een probleem waarbij het voorkomen van veel occluderende instanties het moeilijk maakt om te segmenteren. Aangezien het probleem niet kan opgelost worden met continuïteit alleen, hebben \textcite{Nachmani2020} ook een op identificatie gebaseerde component voor het verlies aan constantheid toegevoegd. Dit werd bereikt door het toevoegen van een nieuw recurrent blok, dat bestaat uit een combinatie van twee bidirectionele RNN's en een skip connectie, het gebruik van verschillende verliezen en een spraakconstantieterm \autocite{Nachmani2020}.

%---------- Methodologie ------------------------------------------------------
%TODO tijdsinschatting maken
\section{Methodologie}%
\label{sec:methodologie}

Het onderzoek zal bestaan uit een literatuurstudie en het onderzoeken en trainen van verschillende modellen. Het grootste deel van de literatuurstudie zal bestaan uit het opzoeken van reeds bestaande modellen en kijken of deze al dan niet toepasbaar zijn op dit onderzoek.

Na de literatuurstudie, zullen eerst modellen getraind worden op audiofragmenten waar amper tot geen ruis aanwezig is. Eens er een model gevonden is dat erin slaagt om stemmen te scheiden op deze audiofragmenten, zal het model getraind worden met audio waar opzettelijk ruis aan toegevoegd is. Uiteindelijk zal het model gebruik maken van de audiofragmenten die reeds beschikbaar zijn door voorgaande bachelorproeven.

Tot slot zal er naar een manier gezocht worden om dit model te integreren in de reeds bestaande applicatie zodat deze het model kan gebruiken om enkel op de stem van de zorgverlener (in opleiding) te focussen voor de speech-to-text.

%---------- Verwachte resultaten ----------------------------------------------
%TODO verbeteren en verder uitwerken
\section{Verwacht resultaat, conclusie}%
\label{sec:verwachte_resultaten}

Het verwachte resultaat van de literatuurstudie is een aantal modellen die getest kunnen worden en eventueel gebruikt kunnen worden om een nieuw model te maken indien blijkt dat de bestaande modellen niet voldoende zijn.

Het verwacht resultaat van het trainen met audiofragmenten zonder ruis is een model dat succesvol stemmen kan scheiden en verder kan verfijnd worden voor audiofragmenten met een minder goede kwaliteit.

Het verwacht resultaat van het trainen met audiofragmenten met ruis is een model dat succesvol stemmen kan scheiden in de test audiofragmenten, maar ook in de fragmenten die reeds aanwezig zijn in het zorglab door voorgaande bachelorproeven.

Het verwachte eindresultaat van deze bachelorproef is een succesvolle integratie van het getrainde model in de reeds bestaande applicatie waardoor nu de keuze bestaat om enkel te focussen op de stem van de zorgverlener.


\printbibliography[heading=bibintoc]



\end{document}