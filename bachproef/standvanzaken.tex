\chapter{\IfLanguageName{dutch}{Stand van zaken}{State of the art}}%
\label{ch:stand-van-zaken}

% Tip: Begin elk hoofdstuk met een paragraaf inleiding die beschrijft hoe
% dit hoofdstuk past binnen het geheel van de bachelorproef. Geef in het
% bijzonder aan wat de link is met het vorige en volgende hoofdstuk.

% Pas na deze inleidende paragraaf komt de eerste sectiehoofding.

Dit hoofdstuk geeft een overzicht over de stand van zaken van zowel de applicatie die reeds bestaat in het Zorglab, als de mogelijkheden en modellen die beschikbaar zijn voor het onderscheiden van stemmen in audio.
Het eerste deel van dit hoofdstuk beschrijft wat studenten van voorgaande jaren al bereikt hebben wat betreft de applicatie. De daaropvolgende delen geven respectievelijk inzicht in de techniek die gebruikt zal worden tijdens het onderzoek en een overzicht van de beschikbare modellen die gebruikt kunnen worden. 

\section{Voorgaand onderzoek in verband met de applicatie}%
%TODO: meer uitwerken en papers van studenten effectief eens lezen
In voorgaande bachelorproeven werd reeds een applicatie ontwikkeld met als doel deze te gebruiken in de opleiding verpleegkunde. Volgens \textcite{Govaerts2022} is het detecteren van elderspeak mogelijk met behulp van PRAAT en Natural Language Processing (NLP). PRAAT is een computerprogramma voor het analyseren, synthetiseren en manipuleren van spraak en werd ontwikkeld sinds 1992 door Paul Boersma en David Weenink \autocite{Govaerts2022}.

Datzelfde jaar werd door \textcite{Gussem2022} geconcludeerd dat elderspeak gedetecteerd kan worden met AI. De software die hiervoor beschikbaar is, is echter gelimiteerd en niet altijd accuraat en kan de stem van de zorgvrager niet wegfilteren voor de analyse \autocite{Gussem2022}.

Later onderzocht \textcite{Daems2023} hoe de detectie van elderspeak verbeterd kan worden door het toepassen van een ruisfilter en een stiltefilter. De conclusie omtrent de ruisfilter was dat deze zeker helpt, maar niet alle ruis uit het signaal kan verwijderen \autocite{Daems2023}. Om dit te verbeteren opperde \textcite{Daems2023} dat het gebruik van meerdere microfoons zou kunnen helpen.
Wanneer het gaat over de stiltefilter, besloot \textcite{Daems2023} dat deze succesvol de stiltes kan detecteren en analyseren, wat het mogelijk maakt om stiltes te verwijderen en andere parameters te verbeteren. \textcite{Daems2023} concludeerde echter ook dat factoren zoals geluidskwaliteit, gesproken dialect, linguïstische verschillen en het doel voor de spraakherkenning de kwaliteit ervan ook sterk beïnvloeden.

\textcite{Branden2024} zocht naar verdere mogelijkheden om de accuraatheid van elderspeak detectie te verhogen. Uit dit onderzoek kon geconcludeerd worden dat het gebruik van een POS-tagger ervoor zorgde dat de accuraatheid significant verhoogd werd, namelijk van 92.71\% naar 99.48\% \autocite{Branden2024}. POS-tagging is een NLP techniek waarbij woordsoorten worden toegekend aan woorden of tokens en hierbij ook rekening houdt met de context \autocite{Branden2024}.

Het onderzoek van \textcite{Schryver2024} had als doel om spraak naar tekst om te zetten. Hieruit moest echter de conclusie getrokken worden dat het accuraat omzetten van gesproken naar geschreven tekst nog niet mogelijk is \autocite{Schryver2024}. \textcite{Schryver2024} onderzocht ook of er mogelijkheden zijn om de performantie van de applicatie te kunnen verbeteren aangezien deze momenteel eerder traag is, maar de conclusie was dat dit niet het geval lijkt te zijn.

\section{Onderzoek naar het onderscheiden van stemmen}
%TODO: nog meer info nodig!
Wanneer gezocht wordt naar manieren om stemmen te onderscheiden, is "speaker diarization" een term die vaak vernoemd wordt. Speaker diarization wordt beschreven als het bepalen wie wanneer spreekt in een audio of video opname waarvaan niet geweten is hoe vaak gesproken wordt en hoeveel sprekers aanwezig zijn \textcite{AngueraMiro2012}.

\section{Reeds bestaande modellen}
%TODO: meer frameworks nodig

\subsection{Diarization modellen}


\subsection{Evaluatie van modellen voor diarization}

