%%=============================================================================
%% Methodologie
%%=============================================================================

\chapter{\IfLanguageName{dutch}{Methodologie}{Methodology}}%
\label{ch:methodologie}

%% TODO: In dit hoofstuk geef je een korte toelichting over hoe je te werk bent
%% gegaan. Verdeel je onderzoek in grote fasen, en licht in elke fase toe wat
%% de doelstelling was, welke deliverables daar uit gekomen zijn, en welke
%% onderzoeksmethoden je daarbij toegepast hebt. Verantwoord waarom je
%% op deze manier te werk gegaan bent.
%% 
%% Voorbeelden van zulke fasen zijn: literatuurstudie, opstellen van een
%% requirements-analyse, opstellen long-list (bij vergelijkende studie),
%% selectie van geschikte tools (bij vergelijkende studie, "short-list"),
%% opzetten testopstelling/PoC, uitvoeren testen en verzamelen
%% van resultaten, analyse van resultaten, ...
%%
%% !!!!! LET OP !!!!!
%%
%% Het is uitdrukkelijk NIET de bedoeling dat je het grootste deel van de corpus
%% van je bachelorproef in dit hoofstuk verwerkt! Dit hoofdstuk is eerder een
%% kort overzicht van je plan van aanpak.
%%
%% Maak voor elke fase (behalve het literatuuronderzoek) een NIEUW HOOFDSTUK aan
%% en geef het een gepaste titel.

Dit hoofdstuk bespreekt kort de fasen van het onderzoek. Per fase word kort besproken wat de doelstelling is en welke keuzes gemaakt werden. In Figuur 3.1 wordt een overzicht getoond van de verschillende fasen en hoe deze elkaar opvolgen.

\paragraph{Literatuurstudie en modellen zoeken}
Deze fase bestaat uit het opzoeken van info omtrent diarization en welke bestaande modellen hier al voor zijn. Hieruit worden de modellen gekozen die het meeste potentieel lijken te hebben om tot een goed eindresultaat te komen en wordt gekeken wat de beste manier van aanpakken is.\\
Naast het zoeken naar modellen, wordt ook gezocht naar methoden om deze modellen te evalueren. Dit zorgt ervoor dat er een gegronde keuze kan gemaakt worden om verder te gaan met een specifiek model. Dit zorgt er in een latere fase ook voor dat er gekozen kan worden voor de beste parameters voor het model.\\
De gevonden modellen worden getest aan de hand van een specifiek audiofragment. Afhankelijk van de bruikbaarheid van het model en hoe succesvol het onderscheiden van stemmen reeds is, wordt het model al dan niet gebruikt worden om te trainen.

\paragraph{Model trainen}
In deze fase worden de modellen die in de eerste fase gekozen werden getraind met Vlaamse audiofragmenten. In eerste instantie gebeurt dit met korte, eenvoudige fragmenten die weinig tot geen overlap en ruis bevatten. Hierna wordt voor complexere audio gekozen die meer overlap bevat om in de derde stap te eindigen met audio die zowel overlap als ruis bevat.\\
Bij elk van deze trainingsstappen wordt aan de hand van evaluatiemethoden die gevonden werden in de eerste fase bijgehouden hoe goed elk model scoort.

\paragraph{Finetunen en evalueren}
In deze fase van het onderzoek worden de modellen die getraind werden met elkaar vergeleken aan de hand van de evaluatiescores uit de voorgaande fase. Op basis hiervan wordt het beste model gekozen. Hierna zullen de parameters van het gekozen model geëvalueerd en gefinetuned worden om zo het best mogelijke resultaat te bekomen.

\paragraph{Implementatie in de applicatie}
In deze fase zal het model opgeslagen worden en geïmplementeerd worden in de broncode van de reeds bestaande applicatie. Er zal ook grondig getest worden of het model hierin het beoogde resultaat heeft.

\begin{figure}
	\centering
	\includegraphics[scale=0.75]{./img/verloop.png}
	\caption{Voorstelling van de fasen van de bachelorproef en hun onderlinge volgorde}
\end{figure}